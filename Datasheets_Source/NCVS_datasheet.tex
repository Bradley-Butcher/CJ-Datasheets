\documentclass[letterpaper, 10 pt, conference]{ieeeconf}  % Comment this line out
                                                          % if you need a4paper
%\documentclass[a4paper, 10pt, conference]{ieeeconf}      % Use this line for a4
                                                          % paper

\IEEEoverridecommandlockouts                              % This command is only
                                                          % needed if you want to
                                                          % use the \thanks command
\overrideIEEEmargins

\usepackage{graphicx}
\usepackage{lipsum}  
\usepackage{xcolor}
\usepackage{titlesec}
\usepackage{hyperref}
\hypersetup{
    colorlinks=true,
    linkcolor=blue,
    filecolor=magenta,      
    urlcolor=cyan,
    pdftitle={Overleaf Example},
    pdfpagemode=FullScreen,
    }
\PassOptionsToPackage{unicode}{hyperref}
\PassOptionsToPackage{naturalnames}{hyperref}

\titleformat{\subsection}
{\color{blue}\normalfont\itshape}
{\color{blue}\thesubsection}{1em}{}

\newcommand{\subtitle}[1]{{\\ \small \normalfont \color{purple} #1}}


\graphicspath{ {images/} }

\title{\LARGE \bf
National Crime Victimization Survey (NCVS) \\{\color{blue}Datasheet}
}

\begin{document}


\maketitle
\thispagestyle{empty}
\pagestyle{empty}

%%%%%%%%%%%%%%%%%%%%%%%%%%%%%%%%%%%%%%%%%%%%%%%%%%%%%%%%%%%%%%%%%%%%%%%%%%%%%%%%
\section{Motivation}

\subsection{For what purpose was the dataset created?}

The National Crime Victimization Survey (NCVS) series was designed to achieve four primary objectives:
\begin{enumerate}
    \item To develop detailed information about the victims and consequences of crime
    \item To estimate the number and types of crimes not reported to police
    \item To provide uniform measures of selected types of crime
    \item To permit comparisons over time and types of areas
\end{enumerate}

\subsection{Was there a specific task in mind?}

Assessing levels of victimization.

\subsection{Was there a specific gap that needed to be filled?}

The NCVS began in 1972 and was developed from work done by the National Opinion Research Center and the President's Commission on Law Enforcement and Administration of Justice. A key finding of the survey was the realization that many crimes were not reported to the police. This survey was created to gain understanding of these unreported crimes.

\subsection{Who created the dataset? \subtitle{e.g., which team or research group and on behalf of which entity: e.g., company, institution, organization}}

The survey is administered by the U.S. Census Bureau (under the U.S. Department of Commerce) on behalf of the Bureau of Justice Statistics (under the U.S. Department of Justice)

\subsection{Any other comments?}

None

\section{Uses}

\subsection{Has the dataset been used for any tasks already? \subtitle{If so, please provide a description.}}

The dataset has been used for a range of victimization studies, looking at things such as:

\begin{itemize}
    \item Assessment of crime levels in the United States 
    \item Comparing Victimization across demographics
    \item Comparing Victimization of specific types of crime across demographics
    \item Assessing the \textit{dark figure of crime}
\end{itemize}

Among many other uses.

\subsection{Is there a repository that links to any or all papers or systems that use the dataset? \subtitle{If so, please provide a link or other access point. }}

\href{https://www.icpsr.umich.edu/web/NACJD/series/95/publications}{https://www.icpsr.umich.edu/web/NACJD/series/95/publications}.

\subsection{What (other) tasks could the dataset be used for?}

This dataset can be used for any task which requires an understanding of the level of victimization for specific crimes, with information on demographics of both the victim and offender.

\subsection{Is there anything about the composition of the dataset or the way it was collected and preprocessed/cleaned/labeled that might impact future uses? \subtitle{For example, is there anything that a dataset consumer might need to know to avoid uses that could result in unfair treatment of individuals or groups (e.g., stereotyping, quality of service issues) or other risks or harms (e.g., legal risks, financial harms)? If so, please provide a description. Is there anything a dataset consumer could do to mitigate these risks or harms?}}

Chapter 7 of Estimating the Incidence of Rape and Sexual assault concludes: "The Bureau of Justice Statistics does not provide public information on the edit process in the National Crime Victimization Survey, although processing and editing errors are an important part of any major survey data collection. The lack of transparency about these processes makes it difficult for data users to fully understand the survey's estimates.". There are questions as to whether the rape and sexual assault are underestimated as they do not align with alternative surveys. 

See \href{https://www.ncbi.nlm.nih.gov/books/NBK202273/}{here} for more detail.

\subsection{Any other comments?}

None.

\section{Composition}

\subsection{What do the instances that comprise the dataset represent? \subtitle{For example: documents, photos, people, or companies}}

Instances in NCVS correspond to incident records.

\subsection{Are there multiple types of instances? \subtitle{e.g., movies, users, and ratings; people and interactions between them; nodes and edges. Please provide a description.}}

Yes, there are four types of records:

\begin{enumerate}
    \item Address ID Record
    \item Household Record
    \item Person Record
    \item Incident Record
\end{enumerate}

\subsection{How many instances are there in total? \subtitle{Of each type, if appropriate.}}

Twice each year, data are obtained from a nationally representative sample of roughly 49,000 households comprising about 100,000 persons on the frequency, characteristics, and consequences of criminal victimization in the United States. The number of incidents will vary from year to year.

\subsection{Does the dataset contain all possible instances or is it a sample (not necessarily random) of instances from a larger set? \subtitle{If the dataset is a sample, then what is the larger set? Is the sample representative of the larger set (e.g., geographic coverage)? If so, please describe how this representativeness was validated/verified. If it is not representative of the larger set, please describe why not (e.g., to cover a more diverse range of instances, because instances were withheld or unavailable)}}

The dataset is 100,000 persons sample of the United States. Weights are provided on a person, household and incident level to produce a representative sample of the US.

\subsection{What data does each instance consist of?}

Each instance consists of the following data:

\begin{itemize}
    \item Type of crime
    \item Date of crime
    \item Location of crime
    \item Relationship between victim and offender
    \item Offender characteristics
    \item Actions taken by the victim
    \item Consequences of victimization
    \item Type of property lost
    \item Crime reported
    \item Reasons for reporting/not reporting
    \item Weapons used
    \item Drugs involved
    \item Alcohol involved
    \item Demographic information including:
    \begin{itemize}
        \item Age
        \item Race
        \item Gender
        \item Income
    \end{itemize}
\end{itemize}

\subsection{Is there a label or target associated with each instance? If so, please provide a description. Are relationships between individual instances made explicit (e.g., users’ movie ratings, social network links)? \subtitle{If so, please describe how these relationships are made explicit}}

No.

\subsection{Are there recommended data splits (e.g., training, development/validation, testing)? \subtitle{If so, please provide a description of these splits, explaining the rationale behind them.}}

No.

\subsection{Are there any errors, sources of noise, or redundancies in the dataset? \subtitle{If so, please provide a description.}}

Weights are provided on a person, household and incident level to produce a representative sample of the US. This increases variance but reduces sample bias.

\subsection{Is the dataset self-contained, or does it link to or otherwise rely on external resources? \subtitle{For example: websites, tweets, other datasets)}}

Self-contained.

\subsection{Does the dataset contain data that might be considered confidential? \subtitle{For example: data that is protected by legal privilege or by doctor–patient confidentiality, data that includes the content of individuals’ nonpublic communications. If so, please provide a description.}}

Yes, demographic and income information.

\subsection{Does the dataset contain data that, if viewed directly, might be offensive, insulting, threatening, or might otherwise cause anxiety? \subtitle{If so, please describe why.}}

No.

\subsection{Does the dataset identify any subpopulations (e.g., by age, gender)? \subtitle{If so, please describe how these subpopulations are identified and provide a description of their respective distributions within the dataset.}}

The dataset contains the following demographic information:

\begin{itemize}
    \item Age
    \item Race
    \item Gender
    \item Income
\end{itemize}

\subsection{Is it possible to identify individuals (i.e., one or more natural persons), either directly or indirectly (i.e., in combination with other data) from the dataset? \subtitle{If so, please describe how.}}

No, the data is sufficiently anonymized.

\subsection{Does the dataset contain data that might be considered sensitive in any way? \subtitle{For example: data that reveals race or ethnic origins, sexual orientations, religious beliefs, political opinions or union memberships, or locations; financial or health data; biometric or genetic data; forms of government identification, such as social security numbers; criminal history) If so, please provide a description.}}

The dataset contains the following sensitive information:

\begin{itemize}
    \item Age
    \item Race
    \item Gender
    \item Income
    \item More
\end{itemize}


\subsection{Any other comments?}

None

\section{Collection Process}

\subsection{How was the data associated with each instance acquired?}

The data was acquired from a bi-annual survey.

\subsection{Was the data directly observable (e.g., raw text, movie ratings), reported by subjects (e.g., survey responses), or indirectly inferred/derived from other data (e.g., part-of-speech tags)? \subtitle{If the data was reported by subjects or indirectly inferred/derived from other data, was the data validated/verified? If so, please describe how.}}

The data is collected in a survey, then processed into it's current format. The survey data is not provided.

\subsection{Who was involved in the data collection process and how were they compensated?}

Each year, data are obtained from a nationally representative sample of about 240,000 persons in about 150,000 households. 

\subsection{Over what timeframe was the data collected? Does this timeframe match the creation timeframe of the data associated with the instances (e.g., recent crawl of old news articles)? \subtitle{If not, please describe the timeframe in which the data associated with the instances was created.}}

The data is collected twice a year.

\subsection{Were any ethical review processes conducted (e.g., by an institutional review board)? \subtitle{If so, please provide a description of these review processes, including the outcomes, as well as a link or other access point to any supporting documentation.}}

Unknown.

\subsection{Did you collect the data from the individuals in question directly, or obtain it via third parties or other sources (e.g., websites)?}

Directly via survey.

\subsection{Were the individuals in question notified about the data collection? \subtitle{If so, please describe (or show with screenshots or other information) how notice was provided, and provide a link or other access point to, or otherwise reproduce, the exact language of the notification itself. Did the individuals in question consent to the collection and use of their data? If so, please describe (or show with screenshots or other information) how consent was requested and provided, and provide a link or other access point to, or otherwise reproduce, the exact language to which the individuals consented. No (see previous question).}}

N/A.

\subsection{If consent was obtained, were the consenting individuals provided with a mechanism to revoke their consent in the future or for certain uses? \subtitle{If so, please provide a description, as well as a link or other access point to the mechanism (if appropriate).}}

Unknown

\subsection{Has an analysis of the potential impact of the dataset and its use on data subjects (e.g., a data protection impact analysis) been conducted? \subtitle{If so, please provide a description of this analysis, including the outcomes, as well as a link or other access point to any supporting documentation.}}

Unknown.

\subsection{Any other comments?}

None.

\section{Pre-processing, cleaning, labeling}

\subsection{Was any preprocessing/cleaning/labeling of the data done (e.g., discretization or bucketing, tokenization, part-of-speech tagging, SIFT feature extraction, removal of instances, processing of missing values)? \subtitle{If so, please provide a description. If not, you may skip the remaining questions in this section.}}

Information on pre-processing is not supplied.

\subsection{Was the “raw” data saved in addition to the preprocessed/cleaned/labeled data (e.g., to support unanticipated future uses)? \subtitle{If so, please provide a link or other access point to the “raw” data. }}

It is not supplied.

\subsection{Is the software that was used to preprocess/clean/label the data available? \subtitle{If so, please provide a link or other access point.}}

No.

\section{Distribution}
\subsection{Will the dataset be distributed to third parties outside of the entity (e.g., company, institution, organization) on behalf of which the dataset was created? If so, please provide a description. How will the dataset will be distributed (e.g., tarball on website, API, GitHub)? \subtitle{Does the dataset have a digital object identifier (DOI)?}}

The dataset is hosted at: \href{https://www.icpsr.umich.edu/web/NACJD/series/95}{https://www.icpsr.umich.edu/web/NACJD/series/95}. 

\subsection{When will the dataset be distributed?}

The dataset is distributed annually.

\subsection{Will the dataset be distributed under a copyright or other intellectual property (IP) license, and/or under applicable terms of use (ToU)? \subtitle{If so, please describe this license and/or ToU, and provide a link or other access point to, or otherwise reproduce, any relevant licensing terms or ToU, as well as any fees associated with these restrictions.}}

The dataset is licensed under a Creative Commons Attribution 4.0 International (CC BY 4.0) License.

\subsection{Have any third parties imposed IP-based or other restrictions on the data associated with the instances? \subtitle{If so, please describe these restrictions, and provide a link or other access point to, or otherwise reproduce, any relevant licensing terms, as well as any fees associated with these restrictions.}}

No.

\subsection{Do any export controls or other regulatory restrictions apply to the dataset or to individual instances? \subtitle{If so, please describe these restrictions, and provide a link or other access point to, or otherwise reproduce, any supporting documentation.}}

No.

\subsection{Any other comments?}

None.

\section{Maintenance}

\subsection{Who will be supporting/hosting/maintaining the dataset?}

The dataset is hosted and supported by the Ministry of Justice and the US Census Bureau.

\subsection{How can the owner/curator/manager of the dataset be contacted (e.g., email address)?}

By contacting the US Census Bureau.

\subsection{Will the dataset be updated (e.g., to correct labeling errors, add new instances, delete instances)? If so, please describe how often, by whom, and how updates will be communicated to dataset consumers (e.g., mailing list, GitHub)?}

New versions of the dataset are released yearly.

\subsection{If the dataset relates to people, are there applicable limits on the retention of the data associated with the instances (e.g., were the individuals in question told that their data would be retained for a fixed period of time and then deleted)? \subtitle{If so, please describe these limits and explain how they will be enforced.}}

Unknown.

\subsection{Will older versions of the dataset continue to be supported/hosted/maintained? \subtitle{If so, please describe how. If not, please describe how its obsolescence will be communicated to dataset consumers. }}

Yes. Previous years of the dataset will continue to be hosted by the Ministry of Justice.

\subsection{If others want to extend/augment/build on/contribute to the dataset, is there a mechanism for them to do so? \subtitle{If so, please provide a description.}}

No.

\medskip
 
\bibliographystyle{unsrt}  
\bibliography{sample}

\end{document}
The owners can be contacted at: UCR-NIBRS@fbi.gov