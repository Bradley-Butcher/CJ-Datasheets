\documentclass[letterpaper, 10 pt, conference]{ieeeconf}  % Comment this line out
                                                          % if you need a4paper
%\documentclass[a4paper, 10pt, conference]{ieeeconf}      % Use this line for a4
                                                          % paper

\IEEEoverridecommandlockouts                              % This command is only
                                                          % needed if you want to
                                                          % use the \thanks command
\overrideIEEEmargins

\usepackage{graphicx}
\usepackage{lipsum}  
\usepackage{xcolor}
\usepackage{titlesec}
\usepackage{hyperref}
\hypersetup{
    colorlinks=true,
    linkcolor=blue,
    filecolor=magenta,      
    urlcolor=cyan,
    pdftitle={Overleaf Example},
    pdfpagemode=FullScreen,
    }
\PassOptionsToPackage{unicode}{hyperref}
\PassOptionsToPackage{naturalnames}{hyperref}

\titleformat{\subsection}
{\color{blue}\normalfont\itshape}
{\color{blue}\thesubsection}{1em}{}

\newcommand{\subtitle}[1]{{\\ \small \normalfont \color{purple} #1}}


\graphicspath{ {images/} }

\title{\LARGE \bf
National Incident-based Reporting System (NIBRS) \\{\color{blue}Datasheet}
}

\begin{document}


\maketitle
\thispagestyle{empty}
\pagestyle{empty}

%%%%%%%%%%%%%%%%%%%%%%%%%%%%%%%%%%%%%%%%%%%%%%%%%%%%%%%%%%%%%%%%%%%%%%%%%%%%%%%%
\section{Motivation}

\subsection{For what purpose was the dataset created?}

The NIBRS dataset was created to improve the overall quality of crime data collected by law enforcement. It aims to provide useful statistics to promote constructive discussion, measured planning, and informed policing. Giving context to specific crime problems such as drug/narcotics and sex offenses, as well as issues like animal cruelty, identity theft, and computer hacking. It intends to provide a nationwide view of crime based on the submission of crime information by law enforcement agencies throughout the country, offering law enforcement and the academic community more comprehensive data than ever before available for management, training, planning, and research. 

\subsection{Was there a specific task in mind?}

NIBRS is an extensive dataset, collecting information on all \textit{Group A} police \textit{incidents} from across the United States. Including:

\begin{itemize}
    \item Arson
    \item Assault Offenses
    \item Bribery
    \item Burglary
    \item Counterfeiting / Forgery
    \item Destruction of Property
    \item Embezzlement
    \item Fraud Offenses
    \item Gambling Offenses
    \item Homicide Offenses
    \item Human Trafficking
    \item Kidnapping / Abduction
    \item Larceny / Theft
    \item Prostitution Offenses
    \item Robbery
    \item Sex Offenses
    \item Weapon Law Violations
\end{itemize}

As such, it's potential uses are multi-faceted. It has not been created with a specific task in mind, but as a national centralized repository of police incident data.

\subsection{Was there a specific gap that needed to be filled?}

The FBI has been aggregating and providing U.S. crime statistics through their \textit{uniform crime reporting (UCR) summary reporting system (SRS)} since 1930. Rather than providing a simple tally of crimes, NIBRS was designed to collect far more detailed information, giving it the ability to provide more detailed, accurate, and meaningful insights. Data are collected about when and where crime takes place, what form it takes, and the characteristics of its victims and perpetrators, among other information.

\subsection{Who created the dataset? \subtitle{e.g., which team or research group and on behalf of which entity: e.g., company, institution, organization}}

NIBRS is collected and managed by the Federal Bureau of Investigation (FBI). Data is submitted by participating agencies.


\subsection{Any other comments?}

None

\section{Uses}

\subsection{Has the dataset been used for any tasks already? \subtitle{If so, please provide a description.}}

The dataset has been used in many studies. Including, but not limited to:

\begin{itemize}
    \item Example 1
    \item Example 2
\end{itemize}

See the next question for a study repository.

\subsection{Is there a repository that links to any or all papers or systems that use the dataset? \subtitle{If so, please provide a link or other access point. }}

The Inter-university Consortium for Political and Social Research (ICPSR) provide a non-exhaustive aggregation of NIBRS publications at: \\
\href{https://www.icpsr.umich.edu/web/ICPSR/series/128/publications}{https://www.icpsr.umich.edu/web/ICPSR/series/128/publications}


\subsection{What (other) tasks could the dataset be used for?}

Potential tasks:

\begin{itemize}
    \item Example 1
    \item Example 2
\end{itemize}

\subsection{Is there anything about the composition of the dataset or the way it was collected and preprocessed/cleaned/labeled that might impact future uses? \subtitle{For example, is there anything that a dataset consumer might need to know to avoid uses that could result in unfair treatment of individuals or groups (e.g., stereotyping, quality of service issues) or other risks or harms (e.g., legal risks, financial harms)? If so, please provide a description. Is there anything a dataset consumer could do to mitigate these risks or harms?}}

NIBRS is a collection of incident records, recorded and provided by thousands of police agencies. While NIBRS attempts to enforce standardisation, each agency will have it's own idiosyncrasies in recording. Some agencies do not record ethnicity, or use different units for recording drug quantities, among other differences. It is important to control for these differences when performing analysis on NIBRS.

Incidents are reported \textit{per-offender}. An incident with five offenders will have five separate incidents recording in NIBRS. There is no direct manner to connect these, so counting the same incident multiple times is possible if not careful.

\subsection{Any other comments?}

None

\section{Composition}

\subsection{What do the instances that comprise the dataset represent? \subtitle{For example: documents, photos, people, or companies}}

In NIBRS instances are incidents. These are where an individual has had an encounter with the police, which may or may not lead to an arrest. 

\subsection{Are there multiple types of instances? \subtitle{e.g., movies, users, and ratings; people and interactions between them; nodes and edges. Please provide a description.}}

Incidents are the sole base-units of NIBRS. Though there can be multiple of the same incident if there are multiple offenders. 

\subsection{How many instances are there in total? \subtitle{Of each type, if appropriate.}}

In 2020, there were \textbf{N} incidents recorded in NIBRS. 

** insert bar chart of offenses, etc. **

\subsection{Does the dataset contain all possible instances or is it a sample (not necessarily random) of instances from a larger set? \subtitle{If the dataset is a sample, then what is the larger set? Is the sample representative of the larger set (e.g., geographic coverage)? If so, please describe how this representativeness was validated/verified. If it is not representative of the larger set, please describe why not (e.g., to cover a more diverse range of instances, because instances were withheld or unavailable)}}

NIBRS contain all incidents recorded by \textbf{participating} agencies. Naturally, any incidents recorded by non-participating agencies are not included. Additionally, this is \textbf{not} a record of \textbf{all crime}. Only a subset of crimes are every encountered by police, and a subset of those are recorded as incidents. 

NIBRS contains population coverage information, it can be determined how representative the incidents recorded are of the jurisdiction in which the agency operates.

\subsection{What data does each instance consist of?}

Each instances contains the following information: 

\begin{itemize}
    \item Incident Information
    \begin{itemize}
        \item Incident Date
        \item Incident Hour
        \item Exceptional Clearance
        \item Exceptional Clearance Date
    \end{itemize}
    \item Offense Information
        \begin{itemize}
        \item Offense Codes
        \item Attempted vs. Completed
        \item Offender Suspected Use (of alcohol, drugs, or computers)
        \item Location
        \item Type and Number of Premises Entered
        \item Type of Criminal Activity/Gang Information
        \item Weapon/Force Used
        \item Bias Motivation
    \end{itemize}
    \item Property Information
    \begin{itemize}
        \item Loss Type
        \item Property Description
        \item Value of Property
        \item Date Recovered
        \item Number of Motor Vehicles Stolen/Recovered
        \item Drug Types and Amounts
    \end{itemize}
    \item Victim Information
    \begin{itemize}
        \item Connection to Offenses
        \item Type of Victim
        \item Age/Sex/Race/Ethnicity/Resident Status of Victim
        \item Assault and Homicide Circumstances
        \item Injury Types
        \item Relationships to Offenders
    \end{itemize}
    \item Offender Information
    \begin{itemize}
        \item Age
        \item Sex
        \item Race
        \item Ethnicity
    \end{itemize}
    \item Arrestee Information
    \begin{itemize}
        \item Arrest Date
        \item Type of Arrest
        \item Arrest Offense Code
        \item Arrestee Weapons
        \item Age/Sex/Race/Ethnicity/Resident
    \end{itemize}
    \item Status of Arrestee
    \begin{itemize}
        \item Disposition of Minor
        \item Group B Arrest Information
        \item Type of Arrest
        \item Arrestee Weapons
        \item Age/Sex/Race/Ethnicity/Resident
        \item Disposition of Minor
    \end{itemize}
\end{itemize}



\subsection{Is there a label or target associated with each instance? If so, please provide a description. Are relationships between individual instances made explicit (e.g., users’ movie ratings, social network links)? \subtitle{If so, please describe how these relationships are made explicit}}

There is no set target label, though a few of interest may be:

\begin{itemize}
    \item Type of Arrest
    \item Example 2
\end{itemize}

\subsection{Are there recommended data splits (e.g., training, development/validation, testing)? \subtitle{If so, please provide a description of these splits, explaining the rationale behind them.}}

When splitting data into multiple sets, be aware that the data is a single database that has been compiled from many agencies. If one wishes to test a predictive model, it may be reasonable to split along agency lines, assessing performance on unseen agencies. 

If a temporal model is being used, to predict future offense numbers for example, this advice should be ignored. Instead, it would make sense to have the same agencies across each split, with each split containing a different time segment. 

\subsection{Are there any errors, sources of noise, or redundancies in the dataset? \subtitle{If so, please provide a description.}}

There are a number of fields which are officer estimates, and thus error prone:

\begin{itemize}
    \item Value of Property
    \item Drug Amount
    \item Race
    \item Ethnicity
    
\end{itemize}

\subsection{Is the dataset self-contained, or does it link to or otherwise rely on external resources? \subtitle{For example: websites, tweets, other datasets)}}

The data is self-contained.

\subsection{Does the dataset contain data that might be considered confidential? \subtitle{For example: data that is protected by legal privilege or by doctor–patient confidentiality, data that includes the content of individuals’ nonpublic communications. If so, please provide a description.}}

The data has been sufficiently pseudo-anonymised. 

\subsection{Does the dataset contain data that, if viewed directly, might be offensive, insulting, threatening, or might otherwise cause anxiety? \subtitle{If so, please describe why.}}

The data contains records of crimes, some of which are violent. As the data is tabular, offense should be minimal.

\subsection{Does the dataset identify any subpopulations (e.g., by age, gender)? \subtitle{If so, please describe how these subpopulations are identified and provide a description of their respective distributions within the dataset.}}

The data contains information on race, age, sex and ethnicity. Additionally, it identifies whether the offense committed was a hate crime against any marginilised group, including LGBTQ+.

**insert offender race distribution **.

\subsection{Is it possible to identify individuals (i.e., one or more natural persons), either directly or indirectly (i.e., in combination with other data) from the dataset? \subtitle{If so, please describe how.}}

No.

\subsection{Does the dataset contain data that might be considered sensitive in any way? \subtitle{For example: data that reveals race or ethnic origins, sexual orientations, religious beliefs, political opinions or union memberships, or locations; financial or health data; biometric or genetic data; forms of government identification, such as social security numbers; criminal history) If so, please provide a description.}}

The data contains information on race, age, sex and ethnicity. Additionally, it identifies whether the offense committed was a hate crime against any marginilised group, including LGBTQ+.

\subsection{Any other comments?}

None

\section{Collection Process}

\subsection{How was the data associated with each instance acquired?}

Incident information is collected by and updated by each respective police agency using their own respective systems as the events occur. Once a year, incidents recorded by a participating agency are converted from their format to the NIBRS format, with help from the state UCR program.

\subsection{Was the data directly observable (e.g., raw text, movie ratings), reported by subjects (e.g., survey responses), or indirectly inferred/derived from other data (e.g., part-of-speech tags)? \subtitle{If the data was reported by subjects or indirectly inferred/derived from other data, was the data validated/verified? If so, please describe how.}}

The data is recorded by police officers. This data is quality controlled and validated twice, once by state UCR programs, and again on reception by the NIBRS program. 

\subsection{Who was involved in the data collection process and how were they compensated?}

The local police agencies.

\subsection{Over what timeframe was the data collected? Does this timeframe match the creation timeframe of the data associated with the instances (e.g., recent crawl of old news articles)? \subtitle{If not, please describe the timeframe in which the data associated with the instances was created.}}

The data has been continuous collected since 1988.

\subsection{Were any ethical review processes conducted (e.g., by an institutional review board)? \subtitle{If so, please provide a description of these review processes, including the outcomes, as well as a link or other access point to any supporting documentation.}}

N/A

\subsection{Did you collect the data from the individuals in question directly, or obtain it via third parties or other sources (e.g., websites)?}

N/A

\subsection{Were the individuals in question notified about the data collection? \subtitle{If so, please describe (or show with screenshots or other information) how notice was provided, and provide a link or other access point to, or otherwise reproduce, the exact language of the notification itself. Did the individuals in question consent to the collection and use of their data? If so, please describe (or show with screenshots or other information) how consent was requested and provided, and provide a link or other access point to, or otherwise reproduce, the exact language to which the individuals consented. No (see previous question).}}

N/A

\subsection{If consent was obtained, were the consenting individuals provided with a mechanism to revoke their consent in the future or for certain uses? \subtitle{If so, please provide a description, as well as a link or other access point to the mechanism (if appropriate).}}

N/A

\subsection{Has an analysis of the potential impact of the dataset and its use on data subjects (e.g., a data protection impact analysis) been conducted? \subtitle{If so, please provide a description of this analysis, including the outcomes, as well as a link or other access point to any supporting documentation.}}

N/A

\subsection{Any other comments?}

None.

\section{Pre-processing, cleaning, labeling}

\subsection{Was any preprocessing/cleaning/labeling of the data done (e.g., discretization or bucketing, tokenization, part-of-speech tagging, SIFT feature extraction, removal of instances, processing of missing values)? \subtitle{If so, please provide a description. If not, you may skip the remaining questions in this section.}}

\lipsum[1]

\subsection{Was the “raw” data saved in addition to the preprocessed/cleaned/labeled data (e.g., to support unanticipated future uses)? \subtitle{If so, please provide a link or other access point to the “raw” data. }}

Police agencies will have their local equivalent of the records they send to the NIBRS program, but these cannot be accessed.

\subsection{Is the software that was used to preprocess/clean/label the data available? \subtitle{If so, please provide a link or other access point.}}

No.

\section{Distribution}
\subsection{Will the dataset be distributed to third parties outside of the entity (e.g., company, institution, organization) on behalf of which the dataset was created? If so, please provide a description. How will the dataset will be distributed (e.g., tarball on website, API, GitHub)? \subtitle{Does the dataset have a digital object identifier (DOI)?}}

The dataset is distributed on the FBIs crime explorer website: \href{https://crime-data-explorer.fr.cloud.gov/pages/downloads}{https://crime-data-explorer.fr.cloud.gov/pages/downloads}

\subsection{When will the dataset be distributed?}

The dataset is distributed on an annual basis.

\subsection{Will the dataset be distributed under a copyright or other intellectual property (IP) license, and/or under applicable terms of use (ToU)? \subtitle{If so, please describe this license and/or ToU, and provide a link or other access point to, or otherwise reproduce, any relevant licensing terms or ToU, as well as any fees associated with these restrictions.}}

The dataset is licensed under a Creative Commons Attribution 4.0 International (CC BY 4.0) License.

\subsection{Have any third parties imposed IP-based or other restrictions on the data associated with the instances? \subtitle{If so, please describe these restrictions, and provide a link or other access point to, or otherwise reproduce, any relevant licensing terms, as well as any fees associated with these restrictions.}}

No.

\subsection{Do any export controls or other regulatory restrictions apply to the dataset or to individual instances? \subtitle{If so, please describe these restrictions, and provide a link or other access point to, or otherwise reproduce, any supporting documentation.}}

No.

\subsection{Any other comments?}

None

\section{Maintenance}

\subsection{Who will be supporting/hosting/maintaining the dataset?}

The FBI.

\subsection{How can the owner/curator/manager of the dataset be contacted (e.g., email address)?}

The owners can be contacted at: UCR-NIBRS@fbi.gov

\subsection{Will the dataset be updated (e.g., to correct labeling errors, add new instances, delete instances)? If so, please describe how often, by whom, and how updates will be communicated to dataset consumers (e.g., mailing list, GitHub)?}

The dataset is published annually. Occasionally UCR will publish blocks of years, e.g. 2000-2010.

\subsection{If the dataset relates to people, are there applicable limits on the retention of the data associated with the instances (e.g., were the individuals in question told that their data would be retained for a fixed period of time and then deleted)? \subtitle{If so, please describe these limits and explain how they will be enforced.}}

N/A.

\subsection{Will older versions of the dataset continue to be supported/hosted/maintained? \subtitle{If so, please describe how. If not, please describe how its obsolescence will be communicated to dataset consumers. }}

All previous releases of the NIBRS dataset will remain available on their website.

\subsection{If others want to extend/augment/build on/contribute to the dataset, is there a mechanism for them to do so? \subtitle{If so, please provide a description.}}

N/A

\medskip
 
\bibliographystyle{unsrt}  
\bibliography{sample}

\end{document}
