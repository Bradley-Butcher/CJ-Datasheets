\documentclass[letterpaper, 10 pt, conference]{ieeeconf}  % Comment this line out
                                                          % if you need a4paper
%\documentclass[a4paper, 10pt, conference]{ieeeconf}      % Use this line for a4
                                                          % paper

\IEEEoverridecommandlockouts                              % This command is only
                                                          % needed if you want to
                                                          % use the \thanks command
\overrideIEEEmargins

\usepackage{graphicx}
\usepackage{lipsum}  
\usepackage{xcolor}
\usepackage{titlesec}

\usepackage{hyperref}
\hypersetup{
    colorlinks=true,
    linkcolor=blue,
    filecolor=magenta,      
    urlcolor=cyan,
    pdftitle={Overleaf Example},
    pdfpagemode=FullScreen,
    }
\PassOptionsToPackage{unicode}{hyperref}
\PassOptionsToPackage{naturalnames}{hyperref}

\titleformat{\subsection}
{\color{blue}\normalfont\itshape}
{\color{blue}\thesubsection}{1em}{}

\newcommand{\subtitle}[1]{{\\ \small \normalfont \color{purple} #1}}


\graphicspath{ {images/} }

\title{\LARGE \bf
Uniform Crime Reporting: Summary Reporting System \\{\color{blue}Datasheet}
}

\begin{document}


\maketitle
\thispagestyle{empty}
\pagestyle{empty}

%%%%%%%%%%%%%%%%%%%%%%%%%%%%%%%%%%%%%%%%%%%%%%%%%%%%%%%%%%%%%%%%%%%%%%%%%%%%%%%%
\section{Motivation}

\subsection{For what purpose was the dataset created?}

The Summary Reporting System (SRS) is part of the FBI's Uniform Crime Reporting program. SRS aims to profile a picture of crime in the United States by collecting monthly agency-level counts of reported criminal offenses. These counts are provided by participating police agencies. 

\subsection{Was there a specific task in mind?}

Outside of collecting aggregated crime statistics there is no specific task in mind.

\subsection{Was there a specific gap that needed to be filled?}

The uniform crime reporting program was established in 1930 to serve as a periodic nationwide assessment of reported crimes, which were not available elsewhere in the criminal justice system.

\subsection{Who created the dataset? \subtitle{e.g., which team or research group and on behalf of which entity: e.g., company, institution, organization}}

The dataset is provided by each participating police agency, and compiled as part of the FBI's Uniform Crime Reporting program.

\subsection{Any other comments?}

None

\section{Uses}

\subsection{Has the dataset been used for any tasks already? \subtitle{If so, please provide a description.}}

The dataset has been used for many tasks, a non-exhaustive repository can be found: \\

\href{https://www.icpsr.umich.edu/web/ICPSR/series/57}{https://www.icpsr.umich.edu/web/ICPSR/series/57}

\subsection{Is there a repository that links to any or all papers or systems that use the dataset? \subtitle{If so, please provide a link or other access point. }}

\href{https://www.icpsr.umich.edu/web/ICPSR/series/57}{https://www.icpsr.umich.edu/web/ICPSR/series/57}

\subsection{What (other) tasks could the dataset be used for?}

The dataset can be used for any task which needs a count of the number of crimes reported throughout the United States. 

\subsection{Is there anything about the composition of the dataset or the way it was collected and preprocessed/cleaned/labeled that might impact future uses? \subtitle{For example, is there anything that a dataset consumer might need to know to avoid uses that could result in unfair treatment of individuals or groups (e.g., stereotyping, quality of service issues) or other risks or harms (e.g., legal risks, financial harms)? If so, please provide a description. Is there anything a dataset consumer could do to mitigate these risks or harms?}}

The SRS uses a "hierachy rule" when counting reported crimes. This means only the \textit{worst} crime of a reported incident is counted. It is also important to note that this only includes reported crime, not all crime that occurs.

\subsection{Any other comments?}

None

\section{Composition}

\subsection{What do the instances that comprise the dataset represent? \subtitle{For example: documents, photos, people, or companies}}

As SRS is an aggregate level dataset, instances correspond an individual reporting agency.

\subsection{Are there multiple types of instances? \subtitle{e.g., movies, users, and ratings; people and interactions between them; nodes and edges. Please provide a description.}}

No.

\subsection{How many instances are there in total? \subtitle{Of each type, if appropriate.}}

In 2020, the UCR SRS program includes data collated from more 15,875 city, university and college, county, state, tribal, and federal law enforcement agencies out of a total 18,623 agencies, covering 98\% of the U.S. population.

\subsection{Does the dataset contain all possible instances or is it a sample (not necessarily random) of instances from a larger set? \subtitle{If the dataset is a sample, then what is the larger set? Is the sample representative of the larger set (e.g., geographic coverage)? If so, please describe how this representativeness was validated/verified. If it is not representative of the larger set, please describe why not (e.g., to cover a more diverse range of instances, because instances were withheld or unavailable)}}

Agency participation is voluntary, as such, SRS does not contain all agencies within the United States. 

\subsection{What data does each instance consist of?}

Each instance contains counts of Part I offenses:

\begin{itemize}
    \item Homicide
    \item Rape
    \item Aggravated Assault
    \item Robbery
    \item Burglary
    \item Larceny-theft
    \item Motor-vehicle theft
    \item Arson
    \item Human Trafficking
\end{itemize}

and counts of Part II offenses:

\begin{itemize}
    \item simple assault
    \item curfew offenses and loitering
    \item embezzlement 
    \item forgery and counterfeiting
    \item disorderly conduct
    \item driving under the influence
    \item drug offenses
    \item fraud
    \item gambling
    \item liquor offenses
    \item offenses against the family
    prostitution, public drunkenness, runaways, sex offenses, stolen property, vandalism, vagrancy, and weapons offenses
\end{itemize}

Additionally, information is collected on demographics, such that one can condition on race, age, or sex to get the offense count for that demographic. 

\subsection{Is there a label or target associated with each instance? If so, please provide a description. Are relationships between individual instances made explicit (e.g., users’ movie ratings, social network links)? \subtitle{If so, please describe how these relationships are made explicit}}

No.

\subsection{Are there recommended data splits (e.g., training, development/validation, testing)? \subtitle{If so, please provide a description of these splits, explaining the rationale behind them.}}

No.

\subsection{Are there any errors, sources of noise, or redundancies in the dataset? \subtitle{If so, please provide a description.}}

No.

\subsection{Is the dataset self-contained, or does it link to or otherwise rely on external resources? \subtitle{For example: websites, tweets, other datasets)}}

Self-contained.

\subsection{Does the dataset contain data that might be considered confidential? \subtitle{For example: data that is protected by legal privilege or by doctor–patient confidentiality, data that includes the content of individuals’ nonpublic communications. If so, please provide a description.}}

No.

\subsection{Does the dataset contain data that, if viewed directly, might be offensive, insulting, threatening, or might otherwise cause anxiety? \subtitle{If so, please describe why.}}

No.

\subsection{Does the dataset identify any subpopulations (e.g., by age, gender)? \subtitle{If so, please describe how these subpopulations are identified and provide a description of their respective distributions within the dataset.}}

The counts can be split by race, and and sex.

\subsection{Is it possible to identify individuals (i.e., one or more natural persons), either directly or indirectly (i.e., in combination with other data) from the dataset? \subtitle{If so, please describe how.}}

No.

\subsection{Does the dataset contain data that might be considered sensitive in any way? \subtitle{For example: data that reveals race or ethnic origins, sexual orientations, religious beliefs, political opinions or union memberships, or locations; financial or health data; biometric or genetic data; forms of government identification, such as social security numbers; criminal history) If so, please provide a description.}}

As the data is aggregate level, this does not apply.

\subsection{Any other comments?}

None.

\section{Collection Process}

\subsection{How was the data associated with each instance acquired?}

By submission from a participating police agency.

\subsection{Was the data directly observable (e.g., raw text, movie ratings), reported by subjects (e.g., survey responses), or indirectly inferred/derived from other data (e.g., part-of-speech tags)? \subtitle{If the data was reported by subjects or indirectly inferred/derived from other data, was the data validated/verified? If so, please describe how.}}

The data was submitted by participating police agencies to the UCR program.

\subsection{Who was involved in the data collection process and how were they compensated?}

Participating police agencies.

\subsection{Over what timeframe was the data collected? Does this timeframe match the creation timeframe of the data associated with the instances (e.g., recent crawl of old news articles)? \subtitle{If not, please describe the timeframe in which the data associated with the instances was created.}}

The data has been collected since 1930. Data since 1980 is available on the UCR website.

\subsection{Were any ethical review processes conducted (e.g., by an institutional review board)? \subtitle{If so, please provide a description of these review processes, including the outcomes, as well as a link or other access point to any supporting documentation.}}

N/A

\subsection{Did you collect the data from the individuals in question directly, or obtain it via third parties or other sources (e.g., websites)?}

N/A

\subsection{Were the individuals in question notified about the data collection? \subtitle{If so, please describe (or show with screenshots or other information) how notice was provided, and provide a link or other access point to, or otherwise reproduce, the exact language of the notification itself. Did the individuals in question consent to the collection and use of their data? If so, please describe (or show with screenshots or other information) how consent was requested and provided, and provide a link or other access point to, or otherwise reproduce, the exact language to which the individuals consented. No (see previous question).}}

N/A

\subsection{If consent was obtained, were the consenting individuals provided with a mechanism to revoke their consent in the future or for certain uses? \subtitle{If so, please provide a description, as well as a link or other access point to the mechanism (if appropriate).}}

N/A

\subsection{Has an analysis of the potential impact of the dataset and its use on data subjects (e.g., a data protection impact analysis) been conducted? \subtitle{If so, please provide a description of this analysis, including the outcomes, as well as a link or other access point to any supporting documentation.}}

N/A

\subsection{Any other comments?}

None

\section{Pre-processing, cleaning, labeling}

\subsection{Was any preprocessing/cleaning/labeling of the data done (e.g., discretization or bucketing, tokenization, part-of-speech tagging, SIFT feature extraction, removal of instances, processing of missing values)? \subtitle{If so, please provide a description. If not, you may skip the remaining questions in this section.}}

Incidents are classified according to the hierachy rule by each agency before counting. For further detail view the documentation: \href{https://www.fbi.gov/file-repository/ucr/ucr-srs-user-manual-v1.pdf/view}{https://www.fbi.gov/file-repository/ucr/ucr-srs-user-manual-v1.pdf/view}

\subsection{Was the “raw” data saved in addition to the preprocessed/cleaned/labeled data (e.g., to support unanticipated future uses)? \subtitle{If so, please provide a link or other access point to the “raw” data. }}

The original incident data will be preserved by the relevant police agency.

\subsection{Is the software that was used to preprocess/clean/label the data available? \subtitle{If so, please provide a link or other access point.}}

N/A

\section{Distribution}
\subsection{Will the dataset be distributed to third parties outside of the entity (e.g., company, institution, organization) on behalf of which the dataset was created? If so, please provide a description. How will the dataset will be distributed (e.g., tarball on website, API, GitHub)? \subtitle{Does the dataset have a digital object identifier (DOI)?}}

N/A

\subsection{When will the dataset be distributed?}

The dataset is distributed yearly.

\subsection{Will the dataset be distributed under a copyright or other intellectual property (IP) license, and/or under applicable terms of use (ToU)? \subtitle{If so, please describe this license and/or ToU, and provide a link or other access point to, or otherwise reproduce, any relevant licensing terms or ToU, as well as any fees associated with these restrictions.}}

The dataset is licensed under a Creative Commons Attribution 4.0 International (CC BY 4.0) License.

\subsection{Have any third parties imposed IP-based or other restrictions on the data associated with the instances? \subtitle{If so, please describe these restrictions, and provide a link or other access point to, or otherwise reproduce, any relevant licensing terms, as well as any fees associated with these restrictions.}}

No.

\subsection{Do any export controls or other regulatory restrictions apply to the dataset or to individual instances? \subtitle{If so, please describe these restrictions, and provide a link or other access point to, or otherwise reproduce, any supporting documentation.}}

No.

\subsection{Any other comments?}

None.

\section{Maintenance}

\subsection{Who will be supporting/hosting/maintaining the dataset?}

The FBI UCR program.

\subsection{How can the owner/curator/manager of the dataset be contacted (e.g., email address)?}

The owners can be contacted at: UCR-SRS@fbi.gov

\subsection{Will the dataset be updated (e.g., to correct labeling errors, add new instances, delete instances)? If so, please describe how often, by whom, and how updates will be communicated to dataset consumers (e.g., mailing list, GitHub)?}

The dataset will not be updated for errors, but new counts are provided on an annual basis.

\subsection{If the dataset relates to people, are there applicable limits on the retention of the data associated with the instances (e.g., were the individuals in question told that their data would be retained for a fixed period of time and then deleted)? \subtitle{If so, please describe these limits and explain how they will be enforced.}}

No.

\subsection{Will older versions of the dataset continue to be supported/hosted/maintained? \subtitle{If so, please describe how. If not, please describe how its obsolescence will be communicated to dataset consumers. }}

Yes. Datasets dating back until 1980 are hosted on the UCR website.

\subsection{If others want to extend/augment/build on/contribute to the dataset, is there a mechanism for them to do so? \subtitle{If so, please provide a description.}}

N/A

\medskip
 
\bibliographystyle{unsrt}  
\bibliography{sample}

\end{document}
