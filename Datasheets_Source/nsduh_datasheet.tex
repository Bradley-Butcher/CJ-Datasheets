\documentclass[letterpaper, 10 pt, conference]{ieeeconf}  % Comment this line out
                                                          % if you need a4paper
%\documentclass[a4paper, 10pt, conference]{ieeeconf}      % Use this line for a4
                                                          % paper

\IEEEoverridecommandlockouts                              % This command is only
                                                          % needed if you want to
                                                          % use the \thanks command
\overrideIEEEmargins

\usepackage{graphicx}
\usepackage{lipsum}  
\usepackage{xcolor}
\usepackage{titlesec}
\usepackage{hyperref}
\hypersetup{
    colorlinks=true,
    linkcolor=blue,
    filecolor=magenta,      
    urlcolor=cyan,
    pdftitle={Overleaf Example},
    pdfpagemode=FullScreen,
    }
\PassOptionsToPackage{unicode}{hyperref}
\PassOptionsToPackage{naturalnames}{hyperref}

\titleformat{\subsection}
{\color{blue}\normalfont\itshape}
{\color{blue}\thesubsection}{1em}{}

\newcommand{\subtitle}[1]{{\\ \small \normalfont \color{purple} #1}}


\graphicspath{ {images/} }

\title{\LARGE \bf
National Survey on Drug Use and Health (NSDUH) \\{\color{blue}Datasheet}
}

\begin{document}


\maketitle
\thispagestyle{empty}
\pagestyle{empty}

%%%%%%%%%%%%%%%%%%%%%%%%%%%%%%%%%%%%%%%%%%%%%%%%%%%%%%%%%%%%%%%%%%%%%%%%%%%%%%%%
\section{Motivation}

\subsection{For what purpose was the dataset created?}

The National Survey on Drug Use and Health (NSDUH) was created to provide up-to-date information on tobacco, alcohol, and drug use, mental health and other health-related issues in the United States.

The data provide estimates of substance use and mental illness at the national, state, and substate levels. NSDUH data also help to identify the extent of substance use and mental illness among different subgroups, estimate trends over time, and determine the need for treatment services.

\subsection{Was there a specific task in mind?}

NSDUH was created to support prevention and treatment programs, monitor substance use trends, estimate the need for treatment and inform public health policy.

\subsection{Was there a specific gap that needed to be filled?}

Prior to NSDUH, there was no annual health survey that reported on substance abuse and other health issues.

\subsection{Who created the dataset? \subtitle{e.g., which team or research group and on behalf of which entity: e.g., company, institution, organization}}

The dataset is created by the Substance Abuse and Mental Health Services Administration (SAMSA), with data collection and analysis conducted under contract with RTI International.

\subsection{Any other comments?}

None.

\section{Uses}

\subsection{Has the dataset been used for any tasks already? \subtitle{If so, please provide a description.}}

The dataset has been used in over 1,700 publications since it's establishment in 1979. A repository of which can be found: \\

\href{https://www.icpsr.umich.edu/web/ICPSR/series/64}{https://www.icpsr.umich.edu/web/ICPSR/series/64}

\subsection{Is there a repository that links to any or all papers or systems that use the dataset? \subtitle{If so, please provide a link or other access point. }}

\href{https://www.icpsr.umich.edu/web/ICPSR/series/64}{https://www.icpsr.umich.edu/web/ICPSR/series/64}


\subsection{What (other) tasks could the dataset be used for?}

The NSDUH dataset can be used for any tasks that requires age at first use, as well as lifetime, annual, and past-month use of many drugs (listed in a subsequent section), as well as treatment history. The following demographics are also recorded:

\begin{itemize}
    \item Age
    \item Race
    \item Sex
    \item Educational Level
    \item Employment Status
    \item Income Level
    \item Veteran Status
    \item Population Density
\end{itemize}

etc.

\subsection{Is there anything about the composition of the dataset or the way it was collected and preprocessed/cleaned/labeled that might impact future uses? \subtitle{For example, is there anything that a dataset consumer might need to know to avoid uses that could result in unfair treatment of individuals or groups (e.g., stereotyping, quality of service issues) or other risks or harms (e.g., legal risks, financial harms)? If so, please provide a description. Is there anything a dataset consumer could do to mitigate these risks or harms?}}

The data is self-reported, which must be considered using it for analysis.

Due to the Covid-19 pandemic, collection methodology changed for 2020, making direct comparison difficult. This should be kept in mind when performing analysis.

For other preprocessing steps, see section 2.3.2 and 2.3.3 from the documentation \href{https://www.samhsa.gov/data/sites/default/files/reports/rpt35330/2020NSDUHMethodSummDefs091721.pdf}{here}.

\subsection{Any other comments?}

None

\section{Composition}

\subsection{What do the instances that comprise the dataset represent? \subtitle{For example: documents, photos, people, or companies}}

Instances in the dataset comprise of an individual who took the survey.

\subsection{Are there multiple types of instances? \subtitle{e.g., movies, users, and ratings; people and interactions between them; nodes and edges. Please provide a description.}}

No.

\subsection{How many instances are there in total? \subtitle{Of each type, if appropriate.}}

The dataset is released yearly, normally with around 55,000 samples. 2020 was an exception due to the Covid-19 pandemic, where there we only 30,000 samples.

\subsection{Does the dataset contain all possible instances or is it a sample (not necessarily random) of instances from a larger set? \subtitle{If the dataset is a sample, then what is the larger set? Is the sample representative of the larger set (e.g., geographic coverage)? If so, please describe how this representativeness was validated/verified. If it is not representative of the larger set, please describe why not (e.g., to cover a more diverse range of instances, because instances were withheld or unavailable)}}

The dataset contains all valid instances recorded from the Survey carried out that year.

\subsection{What data does each instance consist of?}

Each data instance includes age at first use, as well as lifetime, annual, and past-month use of the following drugs:

\begin{itemize}
    \item alcohol
    \item marijuana
    \item cocaine (including crack)
    \item hallucinogens
    \item heroin 
    \item inhalants 
    \item tobacco
    \item pain relievers
    \item tranquilizers
    \item stimulants
    \item sedatives
\end{itemize}

Respondents are also asked about personal and family income, health care access and coverage, illegal activities and arrest records, problems resulting from the use of drugs, and perceptions of risks. Demographic data include gender, race, age, ethnicity, educational level, employment status, income level, veteran status, household composition, and population density.

\subsection{Is there a label or target associated with each instance? If so, please provide a description. Are relationships between individual instances made explicit (e.g., users’ movie ratings, social network links)? \subtitle{If so, please describe how these relationships are made explicit}}

No.

\subsection{Are there recommended data splits (e.g., training, development/validation, testing)? \subtitle{If so, please provide a description of these splits, explaining the rationale behind them.}}

No.

\subsection{Are there any errors, sources of noise, or redundancies in the dataset? \subtitle{If so, please provide a description.}}

Errors dealt with in pre-processing see section 2.3.2 and 2.3.3 from the documentation \href{https://www.samhsa.gov/data/sites/default/files/reports/rpt35330/2020NSDUHMethodSummDefs091721.pdf}{here}.

\subsection{Is the dataset self-contained, or does it link to or otherwise rely on external resources? \subtitle{For example: websites, tweets, other datasets)}}

Self-contained.

\subsection{Does the dataset contain data that might be considered confidential? \subtitle{For example: data that is protected by legal privilege or by doctor–patient confidentiality, data that includes the content of individuals’ nonpublic communications. If so, please provide a description.}}

Yes, the data contains confidential medical information on each subject. 

\subsection{Does the dataset contain data that, if viewed directly, might be offensive, insulting, threatening, or might otherwise cause anxiety? \subtitle{If so, please describe why.}}

No.

\subsection{Does the dataset identify any subpopulations (e.g., by age, gender)? \subtitle{If so, please describe how these subpopulations are identified and provide a description of their respective distributions within the dataset.}}

The dataset identifies the following subpopulations:

\begin{itemize}
    \item Age
    \item Race
    \item Sex
    \item Educational Level
    \item Employment Status
    \item Income Level
    \item Veteran Status
    \item Population Density
\end{itemize}

\subsection{Is it possible to identify individuals (i.e., one or more natural persons), either directly or indirectly (i.e., in combination with other data) from the dataset? \subtitle{If so, please describe how.}}

The location information is not specific, sufficient information isn't provided to identify individuals. 

\subsection{Does the dataset contain data that might be considered sensitive in any way? \subtitle{For example: data that reveals race or ethnic origins, sexual orientations, religious beliefs, political opinions or union memberships, or locations; financial or health data; biometric or genetic data; forms of government identification, such as social security numbers; criminal history) If so, please provide a description.}}

Both the health and demographic information could be considered sensitive.

\subsection{Any other comments?}

None.

\section{Collection Process}

\subsection{How was the data associated with each instance acquired?}

The data was acquired by annual survey carried out by the Substance Abuse and Mental Health Services Administration.

\subsection{Was the data directly observable (e.g., raw text, movie ratings), reported by subjects (e.g., survey responses), or indirectly inferred/derived from other data (e.g., part-of-speech tags)? \subtitle{If the data was reported by subjects or indirectly inferred/derived from other data, was the data validated/verified? If so, please describe how.}}

The data was reported by subjects via survey.

\subsection{Who was involved in the data collection process and how were they compensated?}

A scientific random sample of household addresses are selected across the United States. Once selected, no other address be substituted for any reason. This practice is to ensure the NSDUH data represent the many different types of people in the United States. At the end of the completed interview, participants receive \$30 in appreciation for your help.

\subsection{Over what timeframe was the data collected? Does this timeframe match the creation timeframe of the data associated with the instances (e.g., recent crawl of old news articles)? \subtitle{If not, please describe the timeframe in which the data associated with the instances was created.}}

Data is collected on an annual basis, with surveys taking place throughout the year.

\subsection{Were any ethical review processes conducted (e.g., by an institutional review board)? \subtitle{If so, please provide a description of these review processes, including the outcomes, as well as a link or other access point to any supporting documentation.}}

All projects involving human subjects must be approved by our Office of Research Protection, which serves as RTI’s Institutional Review Board (IRB) under federal regulations. This committee looks very closely at the written introduction to the study to
be sure the respondents are being properly informed.

\subsection{Did you collect the data from the individuals in question directly, or obtain it via third parties or other sources (e.g., websites)?}

Data collection was direct.

\subsection{Were the individuals in question notified about the data collection? \subtitle{If so, please describe (or show with screenshots or other information) how notice was provided, and provide a link or other access point to, or otherwise reproduce, the exact language of the notification itself. Did the individuals in question consent to the collection and use of their data? If so, please describe (or show with screenshots or other information) how consent was requested and provided, and provide a link or other access point to, or otherwise reproduce, the exact language to which the individuals consented. No (see previous question).}}

The individuals taking part in the survey must sign consent forms, which ensure respondants are properly informed.

\subsection{If consent was obtained, were the consenting individuals provided with a mechanism to revoke their consent in the future or for certain uses? \subtitle{If so, please provide a description, as well as a link or other access point to the mechanism (if appropriate).}}

Unknown.

\subsection{Has an analysis of the potential impact of the dataset and its use on data subjects (e.g., a data protection impact analysis) been conducted? \subtitle{If so, please provide a description of this analysis, including the outcomes, as well as a link or other access point to any supporting documentation.}}

Unknown.

\subsection{Any other comments?}

None.

\section{Pre-processing, cleaning, labeling}

\subsection{Was any preprocessing/cleaning/labeling of the data done (e.g., discretization or bucketing, tokenization, part-of-speech tagging, SIFT feature extraction, removal of instances, processing of missing values)? \subtitle{If so, please provide a description. If not, you may skip the remaining questions in this section.}}

For preprocessing steps, see section 2.3.2 and 2.3.3 from the documentation \href{https://www.samhsa.gov/data/sites/default/files/reports/rpt35330/2020NSDUHMethodSummDefs091721.pdf}{here}.

\subsection{Was the “raw” data saved in addition to the preprocessed/cleaned/labeled data (e.g., to support unanticipated future uses)? \subtitle{If so, please provide a link or other access point to the “raw” data. }}

SAMHSA will have access to the raw survey responses, but these are not publically available.

\subsection{Is the software that was used to preprocess/clean/label the data available? \subtitle{If so, please provide a link or other access point.}}

No.

\section{Distribution}

\subsection{Will the dataset be distributed to third parties outside of the entity (e.g., company, institution, organization) on behalf of which the dataset was created? If so, please provide a description. How will the dataset will be distributed (e.g., tarball on website, API, GitHub)? \subtitle{Does the dataset have a digital object identifier (DOI)?}}
The dataset is distributed on the SAMSA website \href{https://www.datafiles.samhsa.gov/dataset/national-survey-drug-use-and-health-2019-nsduh-2019-ds0001}{here}.

\subsection{When will the dataset be distributed?}

The dataset is distributed annually, announced by SAMSA on their website.

\subsection{Will the dataset be distributed under a copyright or other intellectual property (IP) license, and/or under applicable terms of use (ToU)? \subtitle{If so, please describe this license and/or ToU, and provide a link or other access point to, or otherwise reproduce, any relevant licensing terms or ToU, as well as any fees associated with these restrictions.}}

The dataset is licensed under a Creative Commons Attribution 4.0 International (CC BY 4.0) License.

\subsection{Have any third parties imposed IP-based or other restrictions on the data associated with the instances? \subtitle{If so, please describe these restrictions, and provide a link or other access point to, or otherwise reproduce, any relevant licensing terms, as well as any fees associated with these restrictions.}}

No.

\subsection{Do any export controls or other regulatory restrictions apply to the dataset or to individual instances? \subtitle{If so, please describe these restrictions, and provide a link or other access point to, or otherwise reproduce, any supporting documentation.}}

No.

\subsection{Any other comments?}

None.

\section{Maintenance}

\subsection{Who will be supporting/hosting/maintaining the dataset?}

The dataset is supported and hosted by SAMSA in collaboration with RTI international.

\subsection{How can the owner/curator/manager of the dataset be contacted (e.g., email address)?}

RTI international, the company which manage the collection and processing can be contacted at: NSDUH-Helpdesk@rti.org

\subsection{Will the dataset be updated (e.g., to correct labeling errors, add new instances, delete instances)? If so, please describe how often, by whom, and how updates will be communicated to dataset consumers (e.g., mailing list, GitHub)?}

Other than annual releases, no.

\subsection{If the dataset relates to people, are there applicable limits on the retention of the data associated with the instances (e.g., were the individuals in question told that their data would be retained for a fixed period of time and then deleted)? \subtitle{If so, please describe these limits and explain how they will be enforced.}}

No.

\subsection{Will older versions of the dataset continue to be supported/hosted/maintained? \subtitle{If so, please describe how. If not, please describe how its obsolescence will be communicated to dataset consumers. }}

Previous year will continue to be hosted.

\subsection{If others want to extend/augment/build on/contribute to the dataset, is there a mechanism for them to do so? \subtitle{If so, please provide a description.}}

No.

\medskip
 
\bibliographystyle{unsrt}  
\bibliography{sample}

\end{document}
The owners can be contacted at: UCR-NIBRS@fbi.gov