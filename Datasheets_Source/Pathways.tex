\documentclass[letterpaper, 10 pt, conference]{ieeeconf}  % Comment this line out
                                                          % if you need a4paper
%\documentclass[a4paper, 10pt, conference]{ieeeconf}      % Use this line for a4
                                                          % paper

\IEEEoverridecommandlockouts                              % This command is only
                                                          % needed if you want to
                                                          % use the \thanks command
\overrideIEEEmargins

\usepackage{graphicx}
\usepackage{lipsum}  
\usepackage{xcolor}
\usepackage{titlesec}

\titleformat{\subsection}
{\color{blue}\normalfont\itshape}
{\color{blue}\thesubsection}{1em}{}

\newcommand{\subtitle}[1]{{\\ \small \normalfont \color{purple} #1}}


\graphicspath{ {images/} }

\title{\LARGE \bf
Pathways to Desistance \\{\color{blue}Datasheet}
}

\begin{document}


\maketitle
\thispagestyle{empty}
\pagestyle{empty}

%%%%%%%%%%%%%%%%%%%%%%%%%%%%%%%%%%%%%%%%%%%%%%%%%%%%%%%%%%%%%%%%%%%%%%%%%%%%%%%%
\section{Motivation}

\subsection{For what purpose was the dataset created?}

``The larger goals of the Pathways study are to improve decision-making by court and social service personnel and to clarify policy debates about alternatives for serious adolescent offenders. We hope to provide juvenile justice professionals and policy-makers with reliable empirical information that can be applied to improve practice, particularly regarding juveniles' competence and culpability, risk for future offending, and amenability to rehabilitation''

\subsection{Was there a specific task in mind?}

``The aims of the investigation are to: identify initial patterns of how serious adolescent offenders stop antisocial activity; describe the role of social context and developmental changes in promoting these positive changes; and
compare the effects of sanctions and interventions in promoting these changes''

\subsection{Was there a specific gap that needed to be filled?}

``Some commentators have questioned whether a separate juvenile justice system is even warranted, given its dismal record at controlling or deterring juvenile crime. This debate is occurring, however, with limited data on either patterns of desistance or escalation among serious adolescent offenders or the effects of interventions and sanctions on trajectories of offending during and after adolescence. Although some studies suggest that most offenders curtail or stop antisocial behavior in late adolescence, this research has relied on very small samples of serious offenders or on very limited measurement of antisocial behavior patterns and developmental change''.

\subsection{Who created the dataset?}

The Pathways to Desistance study grew out of the planning efforts of the MacArthur Foundation Research Network on Adolescent Development and Juvenile Justice. Network activities provided the initial forum for conceptualizing and planning this study. Additional funding from an array of both federal and private agencies supported data collection and other study activities.

https://www.pathwaysstudy.pitt.edu/people.html


\subsection{Any other comments?}

\lipsum[1]

\section{Uses}

\subsection{Has the dataset been used for any tasks already? \subtitle{If so, please provide a description.}}

https://www.ojp.gov/ncjrs/virtual-library/abstracts/pathways-desistance-final-technical-report

%https://www.tandfonline.com/doi/full/10.1080/15564886.2012.713903?casa_token=tXffs6nqlHEAAAAA%3A973VQnP1GUrG_LK_ciXwvAtf4SnXbqPFlnDTjzlRyaRy_zBKTvTM_PbRNHG006paVljub5Eewg

%https://www.google.co.uk/books/edition/Global_Perspectives_on_Desistance/_SLeCwAAQBAJ?hl=en&gbpv=1&dq=%22Pathways+to+Desistance%22&pg=PT103&printsec=frontcover


\subsection{Is there a repository that links to any or all papers or systems that use the dataset?}

Yes. https://www.pathwaysstudy.pitt.edu/publications.html

https://www.icpsr.umich.edu/web/NAHDAP/studies/29961/versions/V2/publications

\subsection{What (other) tasks could the dataset be used for?}



\subsection{Is there anything about the composition of the dataset or the way it was collected and preprocessed/cleaned/labeled that might impact future uses?}



\subsection{Any other comments?}

\lipsum[1]

\section{Composition}

\subsection{What do the instances that comprise the dataset represent? \subtitle{For example: documents, photos, people, or companies}}

\lipsum[1]

\subsection{Are there multiple types of instances? \subtitle{e.g., movies, users, and ratings; people and interactions between them; nodes and edges. Please provide a description.}}

\lipsum[1]

\subsection{How many instances are there in total? \subtitle{Of each type, if appropriate.}}

\lipsum[1]

\subsection{Does the dataset contain all possible instances or is it a sample (not necessarily random) of instances from a larger set?}

Enrollment into the Pathways to Desistance study occurred over a twenty-six month period between November, 2000 and January, 2003.

To be eligible for the study, individuals had to be in Maricopa County, AZ or Philadelphia, PA and:
1. at least 14 years old and under 18 years old at the time of their committing offense
2. found guilty of a serious offense (predominantly felonies, with a few exceptions for some misdemeanor property offenses, sexual assault, or weapons offenses)
3. had to provide informed assent or consent (parent consent was obtained for all youth under the age of 18 at the time of enrollment)


The proportion of male youth found guilty of a drug charge was capped at $15\%$ to avoid an over-representation of drug offenders. All females who met the age and crime criteria were approached for enrollment as were youth being considered for trial in the adult system.

Twenty percent of the youths approached for participation declined.

\subsection{What data does each instance consist of?}

Interview responses.

\subsection{Is there a label or target associated with each instance? If so, please provide a description. Are relationships between individual instances made explicit (e.g., users’ movie ratings, social network links)? \subtitle{If so, please describe how these relationships are made explicit}}

\lipsum[1]

\subsection{Are there recommended data splits (e.g., training, development/validation, testing)? \subtitle{If so, please provide a description of these splits, explaining the rationale behind them.}}

\lipsum[1]

\subsection{Are there any errors, sources of noise, or redundancies in the dataset? \subtitle{If so, please provide a description.}}

\lipsum[1]

\subsection{Is the dataset self-contained, or does it link to or otherwise rely on external resources? \subtitle{For example: websites, tweets, other datasets)}}

\lipsum[1]

\subsection{Does the dataset contain data that might be considered confidential? \subtitle{For example: data that is protected by legal privilege or by doctor–patient confidentiality, data that includes the content of individuals’ nonpublic communications. If so, please provide a description.}}

\lipsum[1]

\subsection{Does the dataset contain data that, if viewed directly, might be offensive, insulting, threatening, or might otherwise cause anxiety? \subtitle{If so, please describe why.}}

\lipsum[1]

\subsection{Does the dataset identify any subpopulations (e.g., by age, gender)? \subtitle{If so, please describe how these subpopulations are identified and provide a description of their respective distributions within the dataset.}}

\lipsum[1]

\subsection{Is it possible to identify individuals (i.e., one or more natural persons), either directly or indirectly (i.e., in combination with other data) from the dataset? \subtitle{If so, please describe how.}}

\lipsum[1]

\subsection{Does the dataset contain data that might be considered sensitive in any way? \subtitle{For example: data that reveals race or ethnic origins, sexual orientations, religious beliefs, political opinions or union memberships, or locations; financial or health data; biometric or genetic data; forms of government identification, such as social security numbers; criminal history) If so, please provide a description.}}

\lipsum[1]

\subsection{Any other comments?}

\lipsum[1]

\section{Collection Process}

\subsection{How was the data associated with each instance acquired?}

\lipsum[1]

\subsection{Was the data directly observable (e.g., raw text, movie ratings), reported by subjects (e.g., survey responses), or indirectly inferred/derived from other data (e.g., part-of-speech tags)? \subtitle{If the data was reported by subjects or indirectly inferred/derived from other data, was the data validated/verified? If so, please describe how.}}

\lipsum[1]

\subsection{Who was involved in the data collection process and how were they compensated?}

\lipsum[1]

\subsection{Over what timeframe was the data collected? Does this timeframe match the creation timeframe of the data associated with the instances (e.g., recent crawl of old news articles)? \subtitle{If not, please describe the timeframe in which the data associated with the instances was created.}}

\lipsum[1]

\subsection{Were any ethical review processes conducted (e.g., by an institutional review board)? \subtitle{If so, please provide a description of these review processes, including the outcomes, as well as a link or other access point to any supporting documentation.}}

\lipsum[1]

\subsection{Did you collect the data from the individuals in question directly, or obtain it via third parties or other sources (e.g., websites)?}

\lipsum[1]

\subsection{Were the individuals in question notified about the data collection? \subtitle{If so, please describe (or show with screenshots or other information) how notice was provided, and provide a link or other access point to, or otherwise reproduce, the exact language of the notification itself. Did the individuals in question consent to the collection and use of their data? If so, please describe (or show with screenshots or other information) how consent was requested and provided, and provide a link or other access point to, or otherwise reproduce, the exact language to which the individuals consented. No (see previous question).}}

\lipsum[1]

\subsection{If consent was obtained, were the consenting individuals provided with a mechanism to revoke their consent in the future or for certain uses? \subtitle{If so, please provide a description, as well as a link or other access point to the mechanism (if appropriate).}}
\lipsum[1]

\subsection{Has an analysis of the potential impact of the dataset and its use on data subjects (e.g., a data protection impact analysis) been conducted? \subtitle{If so, please provide a description of this analysis, including the outcomes, as well as a link or other access point to any supporting documentation.}}
\lipsum[1]

\subsection{Any other comments?}
\lipsum[1]

\section{Pre-processing, cleaning, labeling}

\subsection{Was any preprocessing/cleaning/labeling of the data done (e.g., discretization or bucketing, tokenization, part-of-speech tagging, SIFT feature extraction, removal of instances, processing of missing values)? \subtitle{If so, please provide a description. If not, you may skip the remaining questions in this section.}}

\lipsum[1]

\subsection{Was the “raw” data saved in addition to the preprocessed/cleaned/labeled data (e.g., to support unanticipated future uses)? \subtitle{If so, please provide a link or other access point to the “raw” data. }}

\lipsum[1]

\subsection{Is the software that was used to preprocess/clean/label the data available? \subtitle{If so, please provide a link or other access point.}}

\lipsum[1]

\section{Distribution}
\subsection{Will the dataset be distributed to third parties outside of the entity (e.g., company, institution, organization) on behalf of which the dataset was created? If so, please provide a description. How will the dataset will be distributed (e.g., tarball on website, API, GitHub)? \subtitle{Does the dataset have a digital object identifier (DOI)?}}

\lipsum[1]

\subsection{When will the dataset be distributed?}

\lipsum[1]

\subsection{Will the dataset be distributed under a copyright or other intellectual property (IP) license, and/or under applicable terms of use (ToU)? \subtitle{If so, please describe this license and/or ToU, and provide a link or other access point to, or otherwise reproduce, any relevant licensing terms or ToU, as well as any fees associated with these restrictions.}}

\lipsum[1]

\subsection{Have any third parties imposed IP-based or other restrictions on the data associated with the instances? \subtitle{If so, please describe these restrictions, and provide a link or other access point to, or otherwise reproduce, any relevant licensing terms, as well as any fees associated with these restrictions.}}

\lipsum[1]

\subsection{Do any export controls or other regulatory restrictions apply to the dataset or to individual instances? \subtitle{If so, please describe these restrictions, and provide a link or other access point to, or otherwise reproduce, any supporting documentation.}}

\lipsum[1]

\subsection{Any other comments?}

\section{Maintenance}

\subsection{Who will be supporting/hosting/maintaining the dataset?}

\lipsum[1]

\subsection{How can the owner/curator/manager of the dataset be contacted (e.g., email address)?}

\lipsum[1]

\subsection{Will the dataset be updated (e.g., to correct labeling errors, add new instances, delete instances)? If so, please describe how often, by whom, and how updates will be communicated to dataset consumers (e.g., mailing list, GitHub)?}

\lipsum[1]

\subsection{If the dataset relates to people, are there applicable limits on the retention of the data associated with the instances (e.g., were the individuals in question told that their data would be retained for a fixed period of time and then deleted)? \subtitle{If so, please describe these limits and explain how they will be enforced.}}

\lipsum[1]

\subsection{Will older versions of the dataset continue to be supported/hosted/maintained? \subtitle{If so, please describe how. If not, please describe how its obsolescence will be communicated to dataset consumers. }}

\lipsum[1]

\subsection{If others want to extend/augment/build on/contribute to the dataset, is there a mechanism for them to do so? \subtitle{If so, please provide a description.}}

\lipsum[1]

\medskip
 
\bibliographystyle{unsrt}  
\bibliography{sample}

\end{document}
The owners can be contacted at: UCR-NIBRS@fbi.gov