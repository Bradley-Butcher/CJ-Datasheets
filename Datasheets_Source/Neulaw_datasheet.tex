\documentclass[letterpaper, 10 pt, conference]{ieeeconf}  % Comment this line out
                                                          % if you need a4paper
%\documentclass[a4paper, 10pt, conference]{ieeeconf}      % Use this line for a4
                                                          % paper

\IEEEoverridecommandlockouts                              % This command is only
                                                          % needed if you want to
                                                          % use the \thanks command
\overrideIEEEmargins

\usepackage{graphicx}
\usepackage{lipsum}  
\usepackage{xcolor}
\usepackage{titlesec}

\titleformat{\subsection}
{\color{blue}\normalfont\itshape}
{\color{blue}\thesubsection}{1em}{}

\newcommand{\subtitle}[1]{{\\ \small \normalfont \color{purple} #1}}


\graphicspath{ {images/} }

\title{\LARGE \bf
NeuLaw's Criminal Record Database\footnote{The same datasets is also known as the Center for Science and Law’s Criminal Record Database} \\{\color{blue}Datasheet}
}

\begin{document}


\maketitle
\thispagestyle{empty}
\pagestyle{empty}

%%%%%%%%%%%%%%%%%%%%%%%%%%%%%%%%%%%%%%%%%%%%%%%%%%%%%%%%%%%%%%%%%%%%%%%%%%%%%%%%
\section{Motivation}

\subsection{For what purpose was the dataset created?}

According to the creators of the dataset, the dataset was created ``To allow large-scale, cross-jurisdictional analyses of criminal arrests" and ''enhance many types of research -- for example, identification of high-frequency offenders, measurement of changes in policing strategies, and quantification of legislative efficacy -- giving policy makers the best data upon which to base law enforcement decisions'' \cite{ormachea2015new}.


\subsection{Was there a specific task in mind?}

Recidivism 

\subsection{Was there a specific gap that needed to be filled?}

* UCR and NIBRS don't have unique 

The advantages of this novel dataset include: (1)
individual identifiers allow for recidivism analysis—
albeit only for repeated bookings within the same
jurisdiction, (2) the presence of all the charges allows
for deeper understanding of all crime, not just a subset,
(3) more and different offender-specific variables than
the UCR, (4) the data represent a comprehensive and
growing picture of information available to judges and
prosecutors, (5) more and different disposition-specific
variables, enabling assessment of small variations in
punishment, (6) continual development, as we see the
CRD as a data platform for the research community,
which will collaborate with us to integrate new datasets
from other jurisdictions or other points in the detention
process (e.g., corrections).

\subsection{Who created the dataset?}

The codebook lists Gabe Haarsma, Sasha Davenport, Pablo A. Ormachea & David M. Eagleman as authors. [CITE CODEBOOK]

\subsection{Any other comments?}

Comment on relationship to the company and updated version looking like they are paid 
http://scilaw.org/risk-assessment/

\section{Uses}

\subsection{Has the dataset been used for any tasks already?}

Related papers: 

https://journalofethics.ama-assn.org/article/enabling-individualized-criminal-sentencing-while-reducing-subjectivity-tablet-based-assessment/2016-03

https://journals.sagepub.com/doi/full/10.1177/0011128717722010

https://shsu-ir.tdl.org/handle/20.500.11875/2383

https://journals.sagepub.com/doi/full/10.1177/0011128717748576


\subsection{Is there a repository that links to any or all papers or systems that use the dataset?}

No.

\subsection{What (other) tasks could the dataset be used for?}

*Expand* 

\subsection{Is there anything about the composition of the dataset or the way it was collected and preprocessed/cleaned/labeled that might impact future uses?}

11. The CRD only contains arrest data and not incident-based data, thus providing a picture of crime at the courthouse level. This means that previous stages in the
law enforcement process (e.g., 911 calls, house calls, etc.) could skew the arrests that make it into courthouse databases. In contrast, the UCR includes all
reports to law enforcement, providing a different angle
on criminal activity. 

10. The recidivism analysis allowed by the CRD only
applies for repeated bookings within the same
jurisdiction. This approach will systematically
undercount the true recidivism rate due to relocation.

The CRD does not have victim data, precluding
analysis of, for example, whether ethnicity or age of
victim affects sentencing.

8. Some jurisdictions have more limited data than the
rest. For example, New York City’s records only list the
most serious offense per arrest and do not yet include
an identifier. We are currently working to obtain the
missing data for NYC.

7. While our Broad categorization allows for
comparisons across jurisdictions, our Detailed
categorization does not: the subcategories become
populated only if the jurisdictions’ labels or code
citations provided enough detail.




\subsection{Any other comments?}


\section{Composition}

\subsection{What do the instances that comprise the dataset represent?}

Criminal records. The specific variables varies depending on the jurisdiction as described below.

\subsection{How many instances are there in total?}

Harris County, TX: 3.1 million records, spanning from 1977 to April, 2012. 

New York City, NY: 9.8 million records spanning from 1977 to 2013. 

Miami-Dade County, FL: 5.7 million records spanning from 1971 to 2012.

\subsection{Does the dataset contain all possible instances or is it a sample (not necessarily random) of instances from a larger set?}

2. The database contains no juvenile records, as those
are not included in basic Freedom of Information Act
requests. We note that juvenile is defined differently in
each locale, so 17 year olds are included in Harris
County records whereas only 18 year olds appear in
New York City and Miami-Dade County records.

3. The database does not include sealed or expunged
records, as those are typically removed from the
underlying county databases. It is likely that this
disproportionately affects certain crime types (e.g.,
traffic offenses).

\lipsum[1]

\subsection{What data does each instance consist of?}

In the \textbf{Harris County dataset}, each instance contains Information regarding the:
1. Offense: date, code, name, degree, bond amount at the time of arrest, category, broad category.
2. Defendant: unique ID, race, gender, DOB (mm/yyyy), height, weight, citizenship status.
3. Case: unique case ID, date filed, offense degree, case bond, case status.
4. Attorney: hired or assigned.
5. Grand jury: date, defendant present, jury action code\footnote{see what these... }
6. Disposition: date, plea, disposition (e.g., dismissed).

In the \textbf{New York City dataset}, each instance contains Information regarding the:
1. Offense: month, year.
2. Arrest: county, month, year, charge, crime category, broad crime category.
3. Defendant: race, gender, age at arrest.
3. Disposition: county, month, year, charge, disposition.

In the \textbf{Miami-Dade County dataset}, each instance contains Information regarding the:
1. Arrest: date, code, crime category, broad crime category
2. Case: date filed, date closed, offense degree, trial type (Bench / Jury), case code, case status.
3. Defendant: race, gender, DOB (mm/yyyy).
4. Disposition: code, plea, disposition.


\subsection{Is there a label or target associated with each instance?}

There is not a pre-specified target label. However, disposition is most suitable to be used as a target label. 


\subsection{Are there any errors, sources of noise, or redundancies in the dataset?}

1. The jurisdictions currently represented in the CRD do
not identify offenders of Hispanic descent. To obtain a
better understanding of the demographics, we have
estimated the Hispanic population by last name.

5. All the records in the database were originally
entered by humans. Aside from typographical errors
(which were relatively straightforward to fix), a larger problem is missing data. For example, some fields have become more populated with time. Birth date was not
as commonly entered in some of the earlier records
from the 1970s and 1980s, but becomes more rigorously entered with time 6. The CRD does not contain corrections records, as
most states do not consider those public. Therefore,
while we know each offender’s sentence at the end of
trial or plea bargaining, we cannot know how long an
offender actually served. This is potentially solvable by
marrying CRD data with independently obtained
corrections records, a strategy we are currently
pursuing.

\subsection{Is the dataset self-contained, or does it link to or otherwise rely on external resources? \subtitle{For example: websites, tweets, other datasets)}}

Self-contained. 

\subsection{Does the dataset contain data that might be considered confidential?}

No.

\subsection{Does the dataset contain data that, if viewed directly, might be offensive, insulting, threatening, or might otherwise cause anxiety? \subtitle{If so, please describe why.}}

No.

\subsection{Does the dataset identify any subpopulations (e.g., by age, gender)?}

*ADD distributions*

\subsection{Is it possible to identify individuals (i.e., one or more natural persons), either directly or indirectly (i.e., in combination with other data) from the dataset? \subtitle{If so, please describe how.}}

Possibly, if comparing to other sources such as news articles. Only relevant for cases that attracted media attention. 

\subsection{Does the dataset contain data that might be considered sensitive in any way?}

Criminal history.


\section{Collection Process}

\subsection{How was the data associated with each instance acquired?}

To acquire the underlying data, we dataset creators ``contacted New York
City (New York), Harris County (Houston), and MiamiDade County (Miami), to obtain copies of their criminal records from their justice information management systems. As public records, the data were obtained via Freedom of Information Act requests"

\subsection{Was the data directly observable (e.g., raw text, movie ratings), reported by subjects (e.g., survey responses), or indirectly inferred/derived from other data (e.g., part-of-speech tags)?}

The data was derived from a dataset of criminal records used by respective local authorities. It was not collected for research purposes. 

\subsection{Who was involved in the data collection process and how were they compensated?}

The data was entered into the courts data systems by employs of the courts. 

\subsection{Over what timeframe was the data collected?}

Harris County, TX -- 1977 to April, 2012. 
New York City, NY -- 1977 to 2013.
Miami-Dade County, FL -- 1971 to 2012.

\subsection{Were any ethical review processes conducted (e.g., by an institutional review board)?}

Unknown. The dataset creators do state that ``The Institutional Review
Board at Baylor College of Medicine exempted this
release of an anonymized dataset from human subject
research oversight because they consist of publicly
available records''

\subsection{Did you collect the data from the individuals in question directly, or obtain it via third parties or other sources (e.g., websites)?}

Third party (government).

\subsection{Were the individuals in question notified about the data collection?}

It is likely the individuals know of their criminal charges. It is unlikely they know it is being used as part of a research dataset. 

\subsection{Has an analysis of the potential impact of the dataset and its use on data subjects (e.g., a data protection impact analysis) been conducted?}

Unknown. 

\section{Pre-processing, cleaning, labeling}

\subsection{Was any preprocessing/cleaning/labeling of the data done (e.g., discretization or bucketing, tokenization, part-of-speech tagging, SIFT feature extraction, removal of instances, processing of missing values)?}

Yes. 

\subsection{Was the “raw” data saved in addition to the preprocessed/cleaned/labeled data (e.g., to support unanticipated future uses)? \subtitle{If so, please provide a link or other access point to the “raw” data. }}

Data processing is described is detail in \cite{ormachea2015new} and in the codebook *ADD citation**. Broadly, the data was cleaned and standardized, and duplicated entries were removed. Entries have been de-identified by removing names, addresses, etc. DOB was replaced with the month and year only. In the Harris County dataset, defendants and cases were given a unique identifiers. The creators added seven calculated variables for all the datasets: 
1. Broad crime category (32 categories), 
2. Detailed crime category ($\sim 150-175$ categories)
3. Standardized disposition\footnote{standardized disposition has 7 possible dispositions: No Action, Dismissal, Transfer, Acquittal, Guilty, Guilty by Plea, Conditional Dismissal, and Unknown/No Final Disposition}
4. Gender, using given name to determine gender when missing or unknown.
5. Race, using surname to add Hispanic ethnicity.
6. The defendant age at the time of case filed or the arrest date.
7. The year the case is filed.
8. Aggregated case numbers to combine multiple offenses into single case (Harris County only).

The calculated age, race and gender variables are added to the dataset alongside the raw variables. 

\subsection{Is the software that was used to preprocess/clean/label the data available?} 

No.

\section{Distribution}
\subsection{Will the dataset be distributed to third parties outside of the entity (e.g., company, institution, organization) on behalf of which the dataset was created? If so, please provide a description. How will the dataset will be distributed (e.g., tarball on website, API, GitHub)?}

The dataset can be found here \cite{Neulaw}.

\subsection{Will the dataset be distributed under a copyright or other intellectual property (IP) license, and/or under applicable terms of use (ToU)?}

The dataset is licensed under a Creative Commons Attribution 4.0 International (CC BY 4.0) License.

\section{Maintenance}

\subsection{Who will be supporting/hosting/maintaining the dataset?}

The dataset is not maintained.

** make a note of the company ** 

http://scilaw.org/risk-assessment/
\medskip
 
\bibliographystyle{unsrt}  
\bibliography{Neulaw_datasheet}

\end{document}
The owners can be contacted at: UCR-NIBRS@fbi.gov